\documentclass[a4paper, 12pt]{report}

\usepackage[top=36pt,bottom=36pt,left=48pt,right=48pt,marginparwidth=110pt,marginparsep=-80pt]{geometry}
\usepackage{marginnote}
\usepackage{enumitem}
\usepackage{relsize}
\usepackage{setspace}
\usepackage{mathtools}
\usepackage{amssymb}
\usepackage{amsthm}

\renewcommand{\thesection}{\arabic{section}}

\def\defeq{\stackrel{\mathrm{def}}{=}}

\theoremstyle{definition}
\newtheorem{define}{Definition}

\begin{document}

\begin{titlepage}
    \begin{doublespace}
    \hspace{0pt}
    \vfill
    \centering
    \Huge
    \textbf{Notes on Analysis}

    \normalsize
    Jonathan E. Fuzaro Alencar
    \vfill
    \hspace{0pt}
    \end{doublespace}
\end{titlepage}

\pagebreak

\chapter{Preliminaries}
\label{ch:preliminaries}

\marginpar{
\begin{center}
\textbf{Greek Alphabet}
\end{center}
\begin{tabular}{lll}
alpha    &  \(\alpha\)         &  \(A\)\\
beta     &  \(\beta\)          &  \(B\)\\
gamma    &  \(\gamma\)         &  \(\Gamma\)\\
delta    &  \(\delta\)         &  \(\Delta\)\\
epsilon  &  \(\epsilon\)       &  \(E\)\\
zeta     &  \(\zeta\)          &  \(Z\)\\
eta      &  \(\eta\)           &  \(H\)\\
iota     &  \(\iota\)          &  \(I\)\\
kappa    &  \(\kappa\)         &  \(K\)\\
lambda   &  \(\lambda\)        &  \(\Lambda\)\\
mu       &  \(\mu\)            &  \(M\)\\
nu       &  \(\nu\)            &  \(N\)\\
xi       &  \(\xi\)            &  \(\Xi\)\\
omicron  &  \(o\)              &  \(O\)\\
pi       &  \(\pi\)            &  \(\Pi\)\\
rho      &  \(\rho\)           &  \(P\)\\
sigma    &  \(\sigma\)         &  \(\Sigma\)\\
tau      &  \(\tau\)           &  \(T\)\\
upsilon  &  \(\upsilon\)       &  \(\Upsilon\)\\
phi      &  \(\phi,\varphi\)   &  \(\Phi\)\\
chi      &  \(\chi\)           &  \(X\)\\
psi      &  \(\psi\)           &  \(\Psi\)\\
omega    &  \(\omega\)         &  \(\Omega\)
\end{tabular}
}

\begin{doublespace}

\section{Sums \& Products}
\label{sec:sum_and_product}

\textbf{Sum Notation}: \(\mathlarger{\sum}\limits_{i=1}^n a_i
= a_1 + a_2 + \dots + a_n\);\\
\
\indent E.g.: \(\sum\limits_{i=1}^3 \sum\limits_{j=2}^4 (i + j)
= (\sum\limits_{j=2}^4 1+j) + (\sum\limits_{j=2}^4 2+j)
+ (\sum\limits_{j=2}^4 3+j)\)\\
\
\textbf{Product Notation}: \(\mathlarger{\prod}\limits_{i=1}^n a_i
= a_1 \cdot a_2 \cdots a_n\);\\
\
\indent E.g.: \(\prod\limits_{i=1}^n i = n!\)

\section{Logic}
\label{sec:logic}

\begin{tabular}{lcl}
\textbf{Universal Quantifier}: & \(\forall\) & ``for all''\\

\textbf{Existential Quantifier}: & \(\exists\) & ``there exists''\\

\textbf{Uniqueness Quantifier}: & \(\exists!\) & ``there exists one and only
one''\\

\textbf{Material Implication}: & \(\implies\) & ``implies''\\

\textbf{Material Equivalence}: & \(\iff \) & ``if and only if (iff)''\\

\textbf{Logical Complement}: & \(\lnot\) & ``not''\\

\textbf{Logical Conjuction}: & \(\land\) & ``and''\\

\textbf{Logical Disjunction}: & \(\lor\) & ``or''\\

\textbf{Exclusive Disjucntion}: & \(\oplus\) & ``exclusive or (xor)''
\end{tabular}

\pagebreak

\section{Set Theory}
\label{sec:set_theory}

\subsection{Notation}

\begin{tabular}{lcl}

\textbf{Set Relations}:
& \(\{a, b\}\) &
``the set containing the elements \(a\) and \(b\)''\\

& \(\{a \mid p(a)\}\) &
``the set of \(a\) such that \(p(a)\) is true''\\

& \(\in\) & ``is an element of'';
e.g. \(3 \in \{1, 2, 3\}\)\\

& \(\notin\) & ``is not an element of'';
e.g. \(4 \notin \{1, 2, 3\}\)\\

& \(=\) & ``equals'';
\((A = B) \iff \forall x \mid (x \in A \iff x \in B)\)\\

& \(\neq\) & ``does not equal''; \(A \neq B \iff \lnot (A = B)\)\\

& \(\subseteq\) & ``is a subset of'';
\((A \subseteq B) \iff \forall x \mid (x \in A \implies x \in B)\)\\

& \(\subset, \subsetneq\) & ``is a proper subset of'';
\((A \subset B) \iff (A \subseteq B \land A \neq B)\)\\

& \(\supseteq\) & ``is a superset of'';
\(A \supseteq B \iff B \subseteq A\)\\

& \(\supset, \supsetneq\) & ``is a proper superset of'';
\(A \supset B \iff (B \supseteq A \land B \neq A)\)\\

\textbf{Set Operations:}
& \(\overline{A}\) & ``the complement of A'';
\(\overline{A} := \{x \mid x \notin A\}\)\\

& \(\cap\) & ``intersect'';
\(A \cap B := \{x \mid x \in A \land x \in B\}\)\\

& \(\cup\) & ``union'';
\(A \cup B := \{x \mid x \in A \lor x \in B\}\)\\

& \(\times\) & ``cross'';
\(A \times B := \{(x, y) \mid x \in A \land y \in B\}\)\\

& \(-, \setminus \) & ``complement, slash'';
\(A - B := \{x \mid x \in A \land x \notin B\}\)\\

\end{tabular}

\subsection{Indexed Operations}

If \(S_i\) is a collection of sets indexed by \(i \in I\) then,

\textbf{Indexed Intersetion:} \(\mathlarger{\bigcap}_{i \in I}S_i
:= S_{I_1} \cap S_{I_2} \cap ...
= \{x \mid \forall i \in I; x \in S_i\}\)

\

\textbf{Indexed Union:} \(\mathlarger{\bigcup}_{i \in I}S_i
:= S_{I_1} \cup S_{I_2} \cup ...
:= \{x \mid \exists i \in I; x \in S_i\}\)


\subsection{Common Sets}
\begin{tabular}{lll}

\(\varnothing\) & the empty set & \(\{\}\) \\

\(\mathbb{N}\) & the \textit{natural numbers} & \(\{1, 2, 3, ...\}\)\\

\(\mathbb{Z}\) & the \textit{integers} & \(\{..., -1, 0, 1, ...\}\)\\

\(\mathbb{Q}\) & the \textit{rational numbers} &
\(\{p/q \mid p,q \in \mathbb{Z} \land q \neq 0\}\)\\

\(\mathbb{R}\) & the \textit{real numbers} & infinite decimals\\

\(\mathbb{C}\) & the \textit{complex numbers} &
\(\{i = \sqrt{-1}, a + ib \mid a, b \in \mathbb{R}\}\)\\
\end{tabular}

\pagebreak

\section{Functions}

\begin{singlespace}
\begin{define}[\textbf{Function as a rule}]
    A \textit{function} consists of two sets, called the \textit{domain} and the
    \textit{codomain}, and a rule that associates to any element in the domain
    exactly one element in the codomain.
\end{define}

\begin{define}[\textbf{Set theoretic function}]
    A \textit{function} \(f : X \rightarrow Y\) is a subset
    \(\Gamma_f \subseteq X \times Y\) having the property that for every
    \(x \in X\), there exists a unique \(y \in Y\) such that
    \((x,y) \in \Gamma_f\). That is,
    \(\Gamma_f = \{(x, y) \mid (\forall x \in X) (\exists! y \in Y)\}\).
\end{define}

\begin{define}[\textbf{Image}]
    The set of all values of \(f\) is called its \textit{image}:
    \(y\) is an element of the image of a function \(f: X \rightarrow Y\) if
    there exists an \(x \in X\) such that \(f(x) = y\).
    The image of \(f: X \rightarrow Y\) is written \(f(X)\);
    it is a subset of Y.
    That is, the image of \(f\) is
    \(\{f(x) \mid x \in X\} = f(X)\).
\end{define}
    
\end{singlespace}

\end{doublespace}
\end{document}
