\documentclass[../Aluffi_Algebra.tex]{subfiles}

\begin{document}

\begin{defn}
    Let \(a, b \in \Z\). We say that `\(b\) divides \(a\)', or `\(b\) is a divisor of \(a\)',
    or `\(a\) is a multiple of \(b\)', and write \(b \mid a\), is an \emph{integer} \(c \in \Z\) such that \(a = bc\).
\end{defn}

\begin{lem}
    If \(b \mid a\) and \(a \neq 0\), then \(\abs{b} \leq \abs{a}\).
\end{lem}

\begin{fact}[Well-ordering Principle]
    Every nonempty set of nonnegative integers contains a least element.
\end{fact}

\begin{thm}[Division with remainder]
    Let \(a, b \in \Z\) with \(b \neq 0\). Then there exists a unique `quotient' \(q \in \Z\) and a 
    unique `remainder' \(r \in \Z\) such that
    \[ a = bq + r \qquad\text{with } \abs{r} < \abs{b}. \]
\end{thm}

\stepcounter{thm}

\begin{defn}
    Let \(a,b \in \Z\). We say that a nonnegative integer \(d\) is the `greatest common divisor' of \(a\) and \(b\),
    denoted \(\gcd(a,b)\) or simply \((a,b)\), if
    \begin{itemize}
        \item \(d \mid a\) and \(d \mid b\); and
        \item if \(c \mid a\) and \(c \mid \)b, then \(c \mid d\).
    \end{itemize}
\end{defn}

\stepcounter{thm}
\stepcounter{thm}

\begin{thm}
    Let \(a, b \in \Z\). Then the greatest common divisor \(d = \gcd(a,b)\) is an integer linear combination of \(a\)
    and \(b\). That is, there exists integers \(m\) and \(n\) such that \(d = ma + nb\).
    \\ \\
    In fact, if \(a\) and \(b\) are not both 0, then \(\gcd(a,b)\) is the smallest positive linear combination
    of \(a\) and \(b\).
\end{thm}

\begin{cor}
    Let \(a, b \in \Z\). Then \(\gcd(a,b) = 1\) if and only if \(1\) may be expressed as a linear combination of \(a\) and \(b\).
\end{cor}

\begin{defn}
    We say that \(a\) and \(b\) are \emph{relatively prime} if \(\gcd(a,b) = 1\).
\end{defn}

\begin{cor}
    Let \(a, b, c \in \Z\). If \(a \mid bc\) and \(\gcd(a,b) = 1\), then \(a \mid c\).
\end{cor}

\stepcounter{thm}
\stepcounter{thm}

\begin{thm}[Euclidean Algorithm]
    Let \(a, b, \in \Z\), with \(b \neq 0\). Then with notation as above,
    \(\gcd(a, b)\) equals the last nonzero remainder \(r_n\).\\
    More explicitely: let \(r_{-2} = a\) and \(r_{-1} = b\); for \(i \geq 0\), let \(r_i\) be the remainder of the
    division of \(r_{i-2}\) by \(r_{i-1}\). Then there is an integer \(n\) such
    that \(r_n \neq 0\) and \(rn_{n+1} = 0\), and \(\gcd(a,b) = r_n\).
\end{thm}

\begin{lem}
    Let \(a, b, q, r \in \Z\), with \(b \neq 0\), and assume that \(a = bq + r\). Then \(\gcd(a,b) = \gcd(b,r)\).
\end{lem}

\stepcounter{thm}

\begin{defn}
    An integer \(p\) is `irreducable' if \(p \neq \pm 1\) and the only divisors of \(p\) are \(\pm 1, \pm p\).\\
    An integer \(\neq 0, \neq \pm 1\) is `reducible' or `composite' if it is not irreducable.
\end{defn}

\begin{lem}
    Assume that \(p\) is an irreducible integer and that \(b\) is not a multiple of \(p\). Then \(b\) and \(p\)
    are relatively prime, that is, \(\gcd(p,b) = 1\).
\end{lem}

\begin{defn}
    An integer \(p\) is `prime' if \(p \neq \pm 1\) and whenver \(p\) divides the product \(bc\) of two integers
    \(b,c\), then \(p \mid b\) or \(p \mid c\).
\end{defn}

\stepcounter{thm}

\begin{thm}
    Let \(p \in \Z, p \neq 0\). Then \(p\) is prime if and only if it is irreducable.
\end{thm}

\begin{thm}[Fundamental Theorem of Arithmetic]
    Every integer \(n \neq 0, \neq \pm 1\) is a product of finitely many irredicubile integers: \(\forall n \in \Z, n
    \neq 0, n \neq \pm 1\), there exists irreducible integers \(q_1, \ldots, q_r\) such that
    \[ n = q_1 \cdots q_r, \]
    \[ n = \prod_r q_r. \]
    Further, this factorization is unique in the sense that if
    \[ n = q_1 \cdots q_r = p_1 \cdots p_s, \]
    with all \(q_i, p_j\) irreducible, then necessarily \(s = r\) and after reordering the factors we have 
    \(p_1 = \pm q_1, p_2 = \pm q_2, \ldots, p_r = \pm q_r\).
\end{thm}

\stepcounter{thm}
\stepcounter{thm}

\begin{prop}
    Let \(a, b \in \Z^{\neq 0}\), and write
    \[a = \pm 2^{\alpha_2} 3^{\alpha_3} 5^{\alpha_5} 7^{\alpha_7} 11^{\alpha_{11} }\cdots,\]
    \[b = \pm 2^{\beta_2}  3^{\beta_3}  5^{\beta_5}  7^{\beta_7}  11^{\beta_{11} }\cdots,\]
    as above. Then the \(\gcd\) of \(a\) and \(b\) is the positive integer
    \[d = 2^{\delta_2} 3^{\delta_3} 5^{\delta_5} 7^{\delta_7} 11^{\delta_11}\cdots,\]
    where \(\delta_i = \min(\alpha_i,\beta_i)\) for all \(i\).
\end{prop}

\stepcounter{thm}

\begin{cor}
    Two nonzero integers \(a, b\) are relatively prime if and only if they have no common irreducible factor.
\end{cor}

\end{document}